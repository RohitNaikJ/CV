\documentclass[letterpaper,11pt]{article}
\newlength{\outerbordwidth}
\pagestyle{empty}
\raggedbottom
\raggedright
\usepackage[svgnames]{xcolor}
\usepackage{framed}
\usepackage{hyperref}
\usepackage{tocloft}
\definecolor{dgray}{gray}{0.4}
\hypersetup{
  colorlinks=true,
  urlcolor=dgray
}
\newcommand{\MYhref}[3][blue]{\href{#2}{\color{#1}{#3}}}%
%-----------------------------------------------------------
%Edit these values as you see fit

\setlength{\outerbordwidth}{3pt}  % Width of border outside of title bars
\definecolor{shadecolor}{gray}{0.75}  % Outer background color of title bars (0 = black, 1 = white)
\definecolor{shadecolorB}{gray}{0.93}  % Inner background color of title bars


%-----------------------------------------------------------
%Margin setup

\setlength{\evensidemargin}{-0.25in}
\setlength{\headheight}{0in}
\setlength{\headsep}{0in}
\setlength{\oddsidemargin}{-0.25in}
\setlength{\paperheight}{11in}
\setlength{\paperwidth}{8.5in}
\setlength{\tabcolsep}{0in}
\setlength{\textheight}{9.5in}
\setlength{\textwidth}{7in}
\setlength{\topmargin}{-0.3in}
\setlength{\topskip}{0in}
\setlength{\voffset}{0.1in}


%-----------------------------------------------------------
%Custom commands
\newcommand{\resitem}[1]{\item #1 \vspace{-2pt}}
\newcommand{\resheading}[1]{\vspace{8pt}
  \parbox{\textwidth}{\setlength{\FrameSep}{\outerbordwidth}
    \begin{shaded}
\setlength{\fboxsep}{0pt}\framebox[\textwidth][l]{\setlength{\fboxsep}{4pt}\fcolorbox{shadecolorB}{shadecolorB}{\textbf{\sffamily{\mbox{~}\makebox[6.762in][l]{\large #1} \vphantom{p\^{E}}}}}}
    \end{shaded}
  }\vspace{-5pt}
}
\newcommand{\ressubheading}[4]{
\begin{tabular*}{6.5in}{l@{\extracolsep{\fill}}r}
    \textbf{#1} & #2 \\
    \textit{#3} & \textit{#4} \\
\end{tabular*}\vspace{-6pt}}
%-----------------------------------------------------------


\begin{document}

\begin{tabular*}{7in}{l@{\extracolsep{\fill}}r}
\textbf{\Large Rohit Naik Jarupla}
& Ph: +91 9810422284 \\
Junior Undergrad, Computer Science,
& \href{mailto:cs1140224@cse.iitd.ernet.in}{cs1140224@cse.iitd.ernet.in}\\
Indian Institute of Technology Delhi
& \href{http://www.cse.iitd.ernet.in/~cs1140224/}{http://www.cse.iitd.ernet.in/$\sim$cs1140224/} \\
\end{tabular*}
\\


%%%%%%%%%%%%%%%%%%%%%%%%%%%%%%
\resheading{Education}
%%%%%%%%%%%%%%%%%%%%%%%%%%%%%%
\ressubheading{Indian Institute of Technology Delhi}{New Delhi, India}{Bachelor of Technology}{2014 - 2018}
\ \\
\ \\
Relevant Courses: Artificial Intelligence, Machine Learning, Operating Systems, Analysis \& Design of Algorithms, Database Management Systems, Computer Networks, Parallel Programming

\ \\
\ressubheading{Hyderabad Public School, Begumpet}{Hyderabad, India}{ICSE Boards}{2002 - 2012}

\ \\
CGPA: 9.47


%%%%%%%%%%%%%%%%%%%%%%%%%%%%%%
\resheading{Work Experience}
%%%%%%%%%%%%%%%%%%%%%%%%%%%%%%

\begin{itemize}

\item
\ressubheading{Web Developer Intern}{NextGen, Bangalore}{Dashboard Development in JavaScript}{Summer 2016}
\begin{itemize}
\item Designed, developed and delivered a large number of customized web based dashboards (linked to the company?s proprietary product named p3 platform) which enabled client companies to monitor their CSR projects. Technology used: HTML5, CSS3, JavaScript, jQuery, Bootstrap, C3.js.
\item Created a library to match the data model of p3 platform with the level of customization required for the web based dashboards. Integrated several libraries to provide for higher level data aggregation and transformation. Technology used: JavaScript, AJAX, JinqJs, Lodash, Loopback powered REST APIs.
\end{itemize}


\end{itemize}


%%%%%%%%%%%%%%%%%%%%%%%%%%%%%%
\resheading{Projects}
%%%%%%%%%%%%%%%%%%%%%%%%%%%%%%
\begin{itemize}

\item
\ressubheading{\MYhref[black]{https://github.com/RohitNaikJ/OS-hohlabs}{Operating System - xv6}}{IIT Delhi}{course project under Prof. Sourav Bansal}{Jan 2017 - Present}
\begin{itemize}
\item This is an ongoing project, in which I build a basic Operating System from scratch. So far, I've implemented basic I/O, co-routines, threads \& non-preemptive \& preemptive scheduling.
\item Written in C and assembly language, x86.
\end{itemize}

\item
\ressubheading{\MYhref[black]{https://github.com/RohitNaikJ/ML-SVMs}{Facial Attractiveness Classifier}}{IIT Delhi}{course project under Prof. Parag Singla}{Feb 2017 - March 2017}
\begin{itemize}
\item Used Support Vector Machines (SVMs) to build a facial attractiveness classifier, using both Linear Model as well as Guassian. Solved the SVM optimization Problem using a general Purpose convex optimization packages, CVXPy and LIBSVM
\end{itemize}

\item
\ressubheading{\MYhref[black]{https://github.com/RohitNaikJ/TAK-Game-Player}{Game Player for TAK}}{IIT Delhi}{course project under Prof. Mausam}{July 2016 - Sep 2016}
\begin{itemize}
\item Designed a bot for the Real Time Strategy Game, TAK, using Adversarial Search.
\item Implemented Depth-Limited MiniMax Tree Search, Alpha-Beta Pruning and Transposition Table. Used Genetic Algorithm to drastically improve the evaluation function. 
\end{itemize}

\item
\ressubheading{Elevator Simulation Model}{IIT Delhi}{course project under Prof. Mausam}{Aug 2016 - Nov 2016}
\begin{itemize}
\item Designed an Elevator Simulator to optimize electricity and total waiting time.
\item Modeled the problem as Markov Decision Process and implemented UCT and function approximation, along with memory optimization due to a huge state space.
\end{itemize}

\item
\ressubheading{\MYhref[black]{https://github.com/RohitNaikJ/Pong-v2.0}{Multiplayer Ping-Pong}}{IIT Delhi}{course project under Prof. Vinay Ribeiro}{March 2016 - May 2016}
\begin{itemize}
\item Wrote server and client programs for a multilayer ping pong game played over the network using Swing Java Library for GUI and Socket Class for networking (UDP Protocol). 
\item Supports up-to 4 players simultaneously and disconnected/crashed player are replaced by bots to ensure continuity of the game.
\end{itemize}

\item
\ressubheading{ARM Processor in VHDL}{IIT Delhi}{course project under Prof. Anshul Kumar}{Feb 2016 - May 2016}
\begin{itemize}
\item Implemented an ARM Processor with RAM, Register File and an ALU, involving Pipe-lined data-path and control-path with data-forwarding and a co-processor for branch prediction
% \item Also implemented an intelligent data-forwarding mechanism and a co-processor for branch prediction.
\end{itemize}

% \item
% \ressubheading{\MYhref[black]{https://github.com/RohitNaikJ/Complaint-System-Django}{Complaint Management System (Android App)}}{IIT Delhi}{course project under Prof. Vinay Ribeiro}{June 2016 - April 2016}
% \begin{itemize}
% \item Portal to lodge, view or mark as resolved, a wide range of Complaints (Institute, Residential or Individual). Back-end Database Programmed in Django.
% \item Complaints are prioritized on votes/comments and are sent to respective concerned Authorites/Departments (Carpentary, Electrician, Plumber, etc.) 
% \end{itemize}

\end{itemize}
%%%%%%%%%%%%%%%%%%%%%%%%%%%%%%
\resheading{Independent Projects}
%%%%%%%%%%%%%%%%%%%%%%%%%%%%%%

\begin{itemize}
\item
\ressubheading{Web Scraper in Python}{}{Python Programming}{July 2016 - Aug 2016}
\begin{itemize}
\item Built a Web Scraping Script in Python using the Modules Beautiful Soup and Selenium that scans a given course page and automatically downloads the Lecture Notes and Tutorials and saves them locally in appropriate format.
\end{itemize}

\item
\ressubheading{Handwritten Digits Recognition Software}{}{Machine Learning}{October 2016}
\begin{itemize}
\item Designed a Neural-Network to decipher images of handwritten digits in MATLAB. Implemented the Back-propagation Algorithm to minimize the Cost Function.
\end{itemize}
\end{itemize}

% %%%%%%%%%%%%%%%%%%%%%%%%%%%%%%
% \resheading{Relevant Courses}
% %%%%%%%%%%%%%%%%%%%%%%%%%%%%%%

% \begin{itemize}
% \item
% \ressubheading{Computer Science}{}{Python Programming}{July 2016 - Aug 2016}
% \begin{itemize}
% \item Artificial Intelligence, Machine Learning, Operating System, Parallel Programming, Analysis \& Design of Algorithms, Database Management Systems, Computer Networks.
% \end{itemize}

% \item
% \ressubheading{Handwritten Digits Recognition Software}{}{Machine Learning}{October 2016}
% \begin{itemize}
% \item Probability \& Stochastic Processes, 
% \end{itemize}
% \end{itemize}



% \begin{itemize}
% \item
% \ressubheading{Computer Science}{IIT Delhi}{course project under Prof. Anshul Kumar}{Feb 2016 - May 2016}
% \begin{itemize}
% \item Artificial Intelligence, Machine Learning, Operating System, Parallel Programming, Analysis & Design of Algorithms, Database Management Systems, Computer Networks
% \end{itemize}

% \item
% \ressubheading{Mathematics & Electrical Engineering}{IIT Delhi}{course project under Prof. Anshul Kumar}{Feb 2016 - May 2016}
% \begin{itemize}
% \item Probability & Stochastic Processes, Signals & systems, Linear Algebra & Differential Equations, Calculus
% \end{itemize}
% \end{itemize}

%%%%%%%%%%%%%%%%%%%%%%%%%%%%%%
\resheading{Awards, Grants \& Honours}
%%%%%%%%%%%%%%%%%%%%%%%%%%%%%%
\begin{tabular*}{7in}{l@{\extracolsep{\fill}}r}
{KVPY National Scholarship}  & \textsc{2012} \\
\emph{Secured All India Rank 5*} \\[5pt]
 {National Science Olympiad 2012}  & \textsc{July 2012} \\
\emph{Secured Rank 69} \\[5pt]
 {Silver Medal for Academic Excellence } & \textsc{March 2012} \\[5pt]
 
\end{tabular*}

%%%%%%%%%%%%%%%%%%%%%%%%%%%%%%
\resheading{Designing and Coding Skills}
%%%%%%%%%%%%%%%%%%%%%%%%%%%%%%

\begin{tabular*}{6in}{l@{\extracolsep{\fill}}l}
{\bf Extensive}
& \textsc{Java, JavaScript, Python, C++, PostgreSQL}\\[2pt]
{\bf Intermediate}
& \textsc{ARM/X86 Assembly, SML, VHDL, Matlab, HTML/CSS}\\[2pt]
{\bf Basic}
& \textsc{Shell Script, PHP} \\[-5pt]
\end{tabular*}

\end{document}
